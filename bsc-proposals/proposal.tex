\documentclass[bibliography=totoc]{article}
\usepackage[utf8]{inputenc}
\usepackage[a4paper]{geometry}
\usepackage{graphicx}
\usepackage{hyperref}
\usepackage[
    hyperref=true,
    url=true,
    isbn=false,
    backref=false,
    style=numeric,
    block=none,
    backend=bibtex,
    doi=true,
    eprint=true,
    sorting=none
]{biblatex}
\bibliography{references.bib}
\AtEveryBibitem{\clearfield{pages}}

\title{Bachelor Thesis Proposal : \\ The Study of Physics Object Measurements using Tag and Probe Method With CMS Open Data}
\author{Siew-Yan Hoh}
\date{\today}

\begin{document}

\maketitle

\section{Executive Summary of the Research Proposal}

Precision measurements of standard model (SM) processes at the Large Hadron Collider (LHC) made tremendous progress in recent years~\cite{ATLAS:2016nqi,CMS:2014pkt,CMS:2013hon,ATLAS:2016rnf,ATLAS:2015iiu}. Notably, the per-mil precision archieved is attributed to the technological advancement of Monte Carlo (MC) generators and the detector calibration technique used to improve the modelling of high-energy hadron collisions and the instrumental systematic effects. Notably, at the center of mass energy, $\sqrt{s} = $ 8 TeV, the archieved energy resolution of photon reconstructed from the CMS (Compact Muon Selenoid) detector is between 1$\%$ and 3$\%$~\cite{2015pho}, making the $H \rightarrow \gamma \gamma$ decay one of the golden decay channels for the Higgs boson discovery~\cite{CMS:2012qbp}.

Under the CMS data policy~\cite{cmsopendata}, the CMS experiment has periodically released research grade datasets (MC and data) to the public domain; allowing scientists outside of the collaboration to study and understand the detector performance, and possibly to exploit the scientific potential of these data. The open data initiative also provide a fertile ground to nurture and train students to perform data analysis on measuring physics processes and physics object performance, thus forging and installing High Energy Physics (HEP) data analysis skill capability among young students in Malaysia.

One of the instrumental systematic effects are originating from the physics objects used in the measurement; the physics objects are reconstructed and identified via several sub-detection system in the CMS detector. Depending on the the type of physics objects, thier reconstruction and identification efficiencies are affected by the background processes associated with it.

This study will demonstrate the application of common HEP data analysis method, the tag and probe technique~\cite{Behnke:1517556}, to study the Physics Objects performance using CMS open data collected at the centre of mass energy $\sqrt{s} =$ 8 TeV, corresponding to the integrated luminosity of 1.8 $fb^{-1}$.

The objectives of the study are to demonstrate the possibility to performing HEP data analysis using CMS open data on the current computational setup; to study the reconstruction and identification efficiencies of the Physics Object; to derive the scale factors and the associated uncertainties to account for the MC inefficiency used to model the data; and to validate and provide those Physics Object's scale factors for full-fledge physics process measurement in the near future.

The expected outcomes of this study are the successful demonstration of the current setup is capable to perform offline HEP data analysis using CMS open data, and to provide outlook on the improvement in term of computational resources for long term HEP study; establishment of HEP workflow to carry out physics object performance study; provide meaningful scale factor for data and MC correction; and to promote young Malaysian students to participate in frontier research such as experimental particle physics.

\section{Detailed Research Proposal}
\subsection{Research Background Study}

bla bla

\subsubsection{Problem Statement}

\subsubsection{Research Significance}

\subsubsection{Research Hypotheses}

\subsubsection{Research Questions}

\subsubsection{Literature Review}

%\bibliographystyle{unsrt}
%\bibliography{Bibliography}
\printbibliography

\subsection{Research Objectives}

The research objectives are:
\begin{enumerate}
  \item bla
  \item bla
\end{enumerate}

\subsection{Research Methodology}

\subsection{Expected Research Outcome}

\begin{enumerate}
  \item New theoretical finding:
  \item Specific application or potential research:
  \item Social economic impact:
\end{enumerate}

\section{Equipment and Materials Access}

\section{Gantt Chart}



\end{document}
